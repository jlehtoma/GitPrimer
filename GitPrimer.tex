\def\handout{0}   % set to 1 to produce 4-up handouts instead of slides
\def\notes{0}     % set to 1 to show \note{}s
\ifnum\handout=1  % see above for an alternative which uses two preamble files
\documentclass[handout,13pt,compress,c]{beamer}
\usepackage{pgfpages}
\pgfpagesuselayout{4 on 1}[letterpaper,landscape,border shrink=4mm]
\setbeamertemplate{footline}[page number]   % omit if don't want slide number at bottom right
\else
\documentclass[13pt,compress,c]{beamer}
\fi
\usetheme{PaloAlto}
\usepackage{graphicx}
\usepackage{natbib}          % for author year citations \citet \citep
\usepackage{relsize}         % for \smaller etc.
\usepackage{xcolor}
\usepackage{listings}
\lstset{language=bash,
        basicstyle=\tiny,
        frame=single,
        backgroundcolor=\color{lightgray},
        commentstyle=\color{black},
        showstringspaces=false
        }

\DeclareGraphicsExtensions{.pdf, .jpg, .png}
\setbeamercolor{normal text}{bg=blue!5}
\setbeamertemplate{footline}[page number] % omit if don't want slide number at bottom right
\ifnum\notes=1 \setbeameroption{show notes} \fi
\newcommand{\ft}[1]{\frametitle{#1}}
\newcommand{\bi}{\begin{itemize}}
\newcommand{\ei}{\end{itemize}}
\newcommand{\figw}[2]{\centerline{\includegraphics[width=#2\textwidth]{#1}}}
\newcommand{\figh}[2]{\centerline{\includegraphics[height=#2\textheight]{#1}}}
\newcommand{\fig}[1]{\centerline{\includegraphics{#1}}}
\newcommand{\foot}[1]{\footnotetext{#1}} % smaller text in bottom margin, e.g. citations
\makeatletter
\renewcommand\@makefntext[1]{\noindent#1} % see p. 114 of LaTeX Companion 2nd edition
\makeatother
\renewcommand\footnoterule{}
\def\newblock{\hskip .11em plus .33em minus .07em}

\title[git \& GitHub]{A brief introduction to git \& GitHub}
\author[Younkin \& Broman]{Samuel G. Younkin, Karl Broman}
\institute{Biostatistics \& Medical Informatics \\ University of Wisconsin-Madison}
\date{October 2013}
%~~~~~~~~~~~~~~~~~~~~~~~~~~~~~~~~~~~~~~~~~~~~~~~~~~~~~~~~~~~~~~~~~~~~~~~~~~
%~~~~~~~~~~~~~~~~~~~~~~~~~~~~~~~~~~~~~~~~~~~~~~~~~~~~~~~~~~~~~~~~~~~~~~~~~~
%~~~~~~~~~~~~~~~~~~~~~~~~~~~~~~~~~~~~~~~~~~~~~~~~~~~~~~~~~~~~~~~~~~~~~~~~~~
\begin{document}
\frame{\titlepage}
\frame{
   \ft{Outline}
\tableofcontents
}
%~~~~~~~~~~~~~~~~~~~~~~~~~~~~~~~~~~~~~~~~~~~~~~~~~~~~~~~~~~~~~~~~~~~~~~~~~~
%~~~~~~~~~~~~~~~~~~~~~~~~~~~~~~~~~~~~~~~~~~~~~~~~~~~~~~~~~~~~~~~~~~~~~~~~~~
%~~~~~~~~~~~~~~~~~~~~~~~~~~~~~~~~~~~~~~~~~~~~~~~~~~~~~~~~~~~~~~~~~~~~~~~~~~
\section{Version control}
\frame{
\ft{Methods for tracking versions}
\bi
\item Don't keep track
\item Save numbered zip files
\item Formal version control
\ei
}

\frame{
\ft{Why use formal version control?}
\bi
\item History of changes
\item Able to go back
\item No worries about breaking things that work
\item Merging changes from multiple people
\ei
}


\section{git}
%~~~~~~~~~~~~~~~~~~~~~~~~~~~~~~~~~~~~~~~~~~~~~~~~~~~~~~~~~~~~~~~~~~~~~~~~~~
%~~~~~~~~~~~~~~~~~~~~~~~~~~~~~~~~~~~~~~~~~~~~~~~~~~~~~~~~~~~~~~~~~~~~~~~~~~
%~~~~~~~~~~~~~~~~~~~~~~~~~~~~~~~~~~~~~~~~~~~~~~~~~~~~~~~~~~~~~~~~~~~~~~~~~~
\frame{
\ft{What is git?}
\bi
\item Version Control System (VCS)
\bi
\item In the context of producing elaborate software to be released to
  the public the need for version control is self-evident
\item But why should we use it?\ei
\item Content Tracking System
\bi
\item Useful for reports, manuscripts, web-sites, as well as
  software
\ei
\ei
}
%~~~~~~~~~~~~~~~~~~~~~~~~~~~~~~~~~~~~~~~~~~~~~~~~~~~~~~~~~~~~~~~~~~~~~~~~~~
%~~~~~~~~~~~~~~~~~~~~~~~~~~~~~~~~~~~~~~~~~~~~~~~~~~~~~~~~~~~~~~~~~~~~~~~~~~
%~~~~~~~~~~~~~~~~~~~~~~~~~~~~~~~~~~~~~~~~~~~~~~~~~~~~~~~~~~~~~~~~~~~~~~~~~~
\frame{
\ft{Why is it named git?}
\bi
\item Created by Linus Torvalds the creator of the linux kernel
\item git: British slang for a foolish or worthless person
\item Linus named the software after himself again
\item Hilarious yes, but perhaps not the best name for such a
  ubiquitous and powerful program \ei }
%~~~~~~~~~~~~~~~~~~~~~~~~~~~~~~~~~~~~~~~~~~~~~~~~~~~~~~~~~~~~~~~~~~~~~~~~~~
%~~~~~~~~~~~~~~~~~~~~~~~~~~~~~~~~~~~~~~~~~~~~~~~~~~~~~~~~~~~~~~~~~~~~~~~~~~
%~~~~~~~~~~~~~~~~~~~~~~~~~~~~~~~~~~~~~~~~~~~~~~~~~~~~~~~~~~~~~~~~~~~~~~~~~~
\frame{
\ft{Why use git?}
\bi
\item history
\item develop without breaking existing package
\item collaboration
\ei
}
%~~~~~~~~~~~~~~~~~~~~~~~~~~~~~~~~~~~~~~~~~~~~~~~~~~~~~~~~~~~~~~~~~~~~~~~~~~
%~~~~~~~~~~~~~~~~~~~~~~~~~~~~~~~~~~~~~~~~~~~~~~~~~~~~~~~~~~~~~~~~~~~~~~~~~~
%~~~~~~~~~~~~~~~~~~~~~~~~~~~~~~~~~~~~~~~~~~~~~~~~~~~~~~~~~~~~~~~~~~~~~~~~~~
\section{git Demo}
%~~~~~~~~~~~~~~~~~~~~~~~~~~~~~~~~~~~~~~~~~~~~~~~~~~~~~~~~~~~~~~~~~~~~~~~~~~
%~~~~~~~~~~~~~~~~~~~~~~~~~~~~~~~~~~~~~~~~~~~~~~~~~~~~~~~~~~~~~~~~~~~~~~~~~~
%~~~~~~~~~~~~~~~~~~~~~~~~~~~~~~~~~~~~~~~~~~~~~~~~~~~~~~~~~~~~~~~~~~~~~~~~~~
\begin{frame}[fragile]
\frametitle{Collaboration}
\bi
\item Sam \& Karl want to collaborate on a presentation
\item In order to do so efficiently the require the following features to their workflow
\bi
%\item Two versions (CK Group \& Faculty)
\item Tracking ability
\bi\item Who made what change and when\ei
\item Backups need to be in place
\bi\item If Sam makes a mistake we must be able to undo it
\item If Karl changes his mind about the direction the project has taken we must be able to revert to any prior state
\ei
\item Publicly available
\bi\item So all may benefit from our insight\ei
\item Annotate changes
\bi\item If Sam makes a change it should be annotated formally so both Karl and future Sam know why it was made\ei
\ei
\item Not surprisingly, git and GitHub together satisfy all of these requirements
\ei
\end{frame}
%~~~~~~~~~~~~~~~~~~~~~~~~~~~~~~~~~~~~~~~~~~~~~~~~~~~~~~~~~~~~~~~~~~~~~~~~~~
%~~~~~~~~~~~~~~~~~~~~~~~~~~~~~~~~~~~~~~~~~~~~~~~~~~~~~~~~~~~~~~~~~~~~~~~~~~
%~~~~~~~~~~~~~~~~~~~~~~~~~~~~~~~~~~~~~~~~~~~~~~~~~~~~~~~~~~~~~~~~~~~~~~~~~~
\begin{frame}[fragile]
\frametitle{Configure}
\bi
\item Configure git to know your name and email address
\ei
\begin{semiverbatim}
\begin{lstlisting}
$ git config --global user.name "Samuel Younkin"
$ git config --global user.email "syounkin@stat.wisc.edu"
\end{lstlisting}
\end{semiverbatim}
\end{frame}
%~~~~~~~~~~~~~~~~~~~~~~~~~~~~~~~~~~~~~~~~~~~~~~~~~~~~~~~~~~~~~~~~~~~~~~~~~~
%~~~~~~~~~~~~~~~~~~~~~~~~~~~~~~~~~~~~~~~~~~~~~~~~~~~~~~~~~~~~~~~~~~~~~~~~~~
%~~~~~~~~~~~~~~~~~~~~~~~~~~~~~~~~~~~~~~~~~~~~~~~~~~~~~~~~~~~~~~~~~~~~~~~~~~
\begin{frame}[fragile]
\frametitle{Initialize}
\bi
\item Create a working directory and initialize it to be a git
  repository
\ei
\begin{semiverbatim}
\begin{lstlisting}
$ mkdir ~/GitPrimer
$ cd ~/GitPrimer
$ git init
Initialized empty Git repository in ~/GitPrimer/.git/
\end{lstlisting}
\end{semiverbatim}
\end{frame}
%~~~~~~~~~~~~~~~~~~~~~~~~~~~~~~~~~~~~~~~~~~~~~~~~~~~~~~~~~~~~~~~~~~~~~~~~~~
%~~~~~~~~~~~~~~~~~~~~~~~~~~~~~~~~~~~~~~~~~~~~~~~~~~~~~~~~~~~~~~~~~~~~~~~~~~
%~~~~~~~~~~~~~~~~~~~~~~~~~~~~~~~~~~~~~~~~~~~~~~~~~~~~~~~~~~~~~~~~~~~~~~~~~~
\begin{frame}[fragile]
\frametitle{Produce Content}
\bi
\item Create a README file
\ei
{\tiny
\begin{semiverbatim}
Welcome to the GitPrimer repository.

Date: Tue Oct  1 14:12:47 CDT 2013

This repository contains source code for a brief git & GitHub tutorial
given by Younkin & Broman at the University of Wisconsin-Madison,
Dept. of Biostatistics & Medical Informatics.

Email Samuel Younkin <syounkin@stat.wisc.edu> with questions or
comments.

\end{semiverbatim}}
\end{frame}
%~~~~~~~~~~~~~~~~~~~~~~~~~~~~~~~~~~~~~~~~~~~~~~~~~~~~~~~~~~~~~~~~~~~~~~~~~~
%~~~~~~~~~~~~~~~~~~~~~~~~~~~~~~~~~~~~~~~~~~~~~~~~~~~~~~~~~~~~~~~~~~~~~~~~~~
%~~~~~~~~~~~~~~~~~~~~~~~~~~~~~~~~~~~~~~~~~~~~~~~~~~~~~~~~~~~~~~~~~~~~~~~~~~
\begin{frame}[fragile]
\frametitle{Incorporate into Repository}
\bi
\item Now we have the file README in our working directory, but it is not in the repository yet
\item We may check the status of the repository and working directory with \texttt{git status}
\ei
\begin{semiverbatim}
\begin{lstlisting}
$ git status
# On branch master
#
# Initial commit
#
# Untracked files:
#   (use "git add <file>..." to include in what will be committed)
#
#README
nothing added to commit but untracked files present (use "git add" to track)
\end{lstlisting}
\end{semiverbatim}

\end{frame}
%~~~~~~~~~~~~~~~~~~~~~~~~~~~~~~~~~~~~~~~~~~~~~~~~~~~~~~~~~~~~~~~~~~~~~~~~~~
%~~~~~~~~~~~~~~~~~~~~~~~~~~~~~~~~~~~~~~~~~~~~~~~~~~~~~~~~~~~~~~~~~~~~~~~~~~
%~~~~~~~~~~~~~~~~~~~~~~~~~~~~~~~~~~~~~~~~~~~~~~~~~~~~~~~~~~~~~~~~~~~~~~~~~~
\begin{frame}[fragile]
\frametitle{Incorporate into Repository}
\bi
\item To include README in the repository we must first add README to the index using \texttt{git add}
\begin{semiverbatim}
\begin{lstlisting}
$ git add README
\end{lstlisting}
\end{semiverbatim}
\ei
\end{frame}
%~~~~~~~~~~~~~~~~~~~~~~~~~~~~~~~~~~~~~~~~~~~~~~~~~~~~~~~~~~~~~~~~~~~~~~~~~~
%~~~~~~~~~~~~~~~~~~~~~~~~~~~~~~~~~~~~~~~~~~~~~~~~~~~~~~~~~~~~~~~~~~~~~~~~~~
%~~~~~~~~~~~~~~~~~~~~~~~~~~~~~~~~~~~~~~~~~~~~~~~~~~~~~~~~~~~~~~~~~~~~~~~~~~
\begin{frame}[fragile]
\frametitle{Incorporate into Repository}
\bi
\item Now, it is in the index, but it is not in the repository until it is ``committed''
\item Now that it has been added to the index it is ``staged'' for a commit
\begin{semiverbatim}
\begin{lstlisting}
$ git status
# On branch master
#
# Initial commit
#
# Changes to be committed:
#   (use ``git rm --cached <file>...'' to unstage)
#
#new file:   README
#
\end{lstlisting}
\end{semiverbatim}
\item Note the clear message indicating how one can unstage the file
\ei
\end{frame}
%~~~~~~~~~~~~~~~~~~~~~~~~~~~~~~~~~~~~~~~~~~~~~~~~~~~~~~~~~~~~~~~~~~~~~~~~~~
%~~~~~~~~~~~~~~~~~~~~~~~~~~~~~~~~~~~~~~~~~~~~~~~~~~~~~~~~~~~~~~~~~~~~~~~~~~
%~~~~~~~~~~~~~~~~~~~~~~~~~~~~~~~~~~~~~~~~~~~~~~~~~~~~~~~~~~~~~~~~~~~~~~~~~~
\begin{frame}[fragile]
\frametitle{Incorporate into Repository}
\bi
\item To commit to the repository we use \texttt{git commit}
\begin{semiverbatim}
\begin{lstlisting}
$ git commit -m "Initial commit of README file"
[master (root-commit) 32c9d01] Initial commit of README file
 1 file changed, 14 insertions(+)
 create mode 100644 README
\end{lstlisting}
\end{semiverbatim}
\item The \texttt{-m} argument allows one to enter a message in the command line
\item Without \texttt{-m} \texttt{git} will spawn a text editor when you commit
\item git does not allow one to make a commit without entering a log message
\ei
\end{frame}
%~~~~~~~~~~~~~~~~~~~~~~~~~~~~~~~~~~~~~~~~~~~~~~~~~~~~~~~~~~~~~~~~~~~~~~~~~~
%~~~~~~~~~~~~~~~~~~~~~~~~~~~~~~~~~~~~~~~~~~~~~~~~~~~~~~~~~~~~~~~~~~~~~~~~~~
%~~~~~~~~~~~~~~~~~~~~~~~~~~~~~~~~~~~~~~~~~~~~~~~~~~~~~~~~~~~~~~~~~~~~~~~~~~
\begin{frame}[fragile]
\frametitle{Incorporate into Repository}
\bi
\item Now that it has been committed it is in the repository 
\item We inspect the state of the repository again with \texttt{git log}
\ei
\begin{semiverbatim}
\begin{lstlisting}
$ git log
commit 32c9d013132948005408823e79216a891efe357c
Author: Samuel Younkin <syounkin@stat.wisc.edu>
Date:   Thu Oct 3 14:05:13 2013 -0500

    Initial commit of README file

\end{lstlisting}
\end{semiverbatim}
\end{frame}
%~~~~~~~~~~~~~~~~~~~~~~~~~~~~~~~~~~~~~~~~~~~~~~~~~~~~~~~~~~~~~~~~~~~~~~~~~~
%~~~~~~~~~~~~~~~~~~~~~~~~~~~~~~~~~~~~~~~~~~~~~~~~~~~~~~~~~~~~~~~~~~~~~~~~~~
%~~~~~~~~~~~~~~~~~~~~~~~~~~~~~~~~~~~~~~~~~~~~~~~~~~~~~~~~~~~~~~~~~~~~~~~~~~
\begin{frame}[fragile]
\frametitle{Incorporate into Repository}
\bi
\item Let's look at the commit log for the repository as it was when I wrote this bullet point
{\footnotesize \begin{semiverbatim}
\begin{lstlisting}
$ git log --pretty=format:"%h - %an, %ar : %s" -10
e946257 - Samuel Younkin, 26 minutes ago : Added *.vrb to .gitignore
9c2b943 - Samuel Younkin, 22 hours ago : Worked on first four sections
311a538 - Karl Broman, 2 days ago : chmod on GitPrimer.tex (not executable)
516108b - Karl Broman, 2 days ago : Add a Makefile
b7fbf7b - Karl Broman, 2 days ago : Remove the Thank you page
c072fbc - Karl Broman, 2 days ago : Simplify institution and split into two lines
173091b - Samuel Younkin, 2 days ago : Fixed time-stamping in README
92a585c - Samuel Younkin, 2 days ago : Initial commit
\end{lstlisting}
\end{semiverbatim}}
\item We include arguments to \texttt{git log} to make the output more readable.
\ei
\end{frame}
%~~~~~~~~~~~~~~~~~~~~~~~~~~~~~~~~~~~~~~~~~~~~~~~~~~~~~~~~~~~~~~~~~~~~~~~~~~
%~~~~~~~~~~~~~~~~~~~~~~~~~~~~~~~~~~~~~~~~~~~~~~~~~~~~~~~~~~~~~~~~~~~~~~~~~~
%~~~~~~~~~~~~~~~~~~~~~~~~~~~~~~~~~~~~~~~~~~~~~~~~~~~~~~~~~~~~~~~~~~~~~~~~~~
\section{GitHub}
%~~~~~~~~~~~~~~~~~~~~~~~~~~~~~~~~~~~~~~~~~~~~~~~~~~~~~~~~~~~~~~~~~~~~~~~~~~
%~~~~~~~~~~~~~~~~~~~~~~~~~~~~~~~~~~~~~~~~~~~~~~~~~~~~~~~~~~~~~~~~~~~~~~~~~~
%~~~~~~~~~~~~~~~~~~~~~~~~~~~~~~~~~~~~~~~~~~~~~~~~~~~~~~~~~~~~~~~~~~~~~~~~~~
\frame{
\ft{What is GitHub?}
\bi
\item Foo!
\ei
}
%~~~~~~~~~~~~~~~~~~~~~~~~~~~~~~~~~~~~~~~~~~~~~~~~~~~~~~~~~~~~~~~~~~~~~~~~~~
%~~~~~~~~~~~~~~~~~~~~~~~~~~~~~~~~~~~~~~~~~~~~~~~~~~~~~~~~~~~~~~~~~~~~~~~~~~
%~~~~~~~~~~~~~~~~~~~~~~~~~~~~~~~~~~~~~~~~~~~~~~~~~~~~~~~~~~~~~~~~~~~~~~~~~~
\frame{
\ft{Why use GitHub?}
\bi
\item Foo!
\ei
}
%~~~~~~~~~~~~~~~~~~~~~~~~~~~~~~~~~~~~~~~~~~~~~~~~~~~~~~~~~~~~~~~~~~~~~~~~~~
%~~~~~~~~~~~~~~~~~~~~~~~~~~~~~~~~~~~~~~~~~~~~~~~~~~~~~~~~~~~~~~~~~~~~~~~~~~
%~~~~~~~~~~~~~~~~~~~~~~~~~~~~~~~~~~~~~~~~~~~~~~~~~~~~~~~~~~~~~~~~~~~~~~~~~~
\section{GitHub Demo}
%~~~~~~~~~~~~~~~~~~~~~~~~~~~~~~~~~~~~~~~~~~~~~~~~~~~~~~~~~~~~~~~~~~~~~~~~~~
%~~~~~~~~~~~~~~~~~~~~~~~~~~~~~~~~~~~~~~~~~~~~~~~~~~~~~~~~~~~~~~~~~~~~~~~~~~
%~~~~~~~~~~~~~~~~~~~~~~~~~~~~~~~~~~~~~~~~~~~~~~~~~~~~~~~~~~~~~~~~~~~~~~~~~~
\frame{
\ft{Clone, Push \& Pull}
\bi
\item Foo!
\ei
}
%~~~~~~~~~~~~~~~~~~~~~~~~~~~~~~~~~~~~~~~~~~~~~~~~~~~~~~~~~~~~~~~~~~~~~~~~~~
%~~~~~~~~~~~~~~~~~~~~~~~~~~~~~~~~~~~~~~~~~~~~~~~~~~~~~~~~~~~~~~~~~~~~~~~~~~
%~~~~~~~~~~~~~~~~~~~~~~~~~~~~~~~~~~~~~~~~~~~~~~~~~~~~~~~~~~~~~~~~~~~~~~~~~~
\section{Learn More}
\frame{
\ft{Resources}
\bi
\item Foo!
\ei
}
%~~~~~~~~~~~~~~~~~~~~~~~~~~~~~~~~~~~~~~~~~~~~~~~~~~~~~~~~~~~~~~~~~~~~~~~~~~
%~~~~~~~~~~~~~~~~~~~~~~~~~~~~~~~~~~~~~~~~~~~~~~~~~~~~~~~~~~~~~~~~~~~~~~~~~~
%~~~~~~~~~~~~~~~~~~~~~~~~~~~~~~~~~~~~~~~~~~~~~~~~~~~~~~~~~~~~~~~~~~~~~~~~~~
\end{document}
